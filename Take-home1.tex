\documentclass{article}
\usepackage[utf8]{inputenc}
\usepackage{blindtext}
\usepackage[T1]{fontenc}
\usepackage{amsmath}
\usepackage{amssymb}
\usepackage{graphicx}
\usepackage{listings}
\usepackage{color}


\title{Take-Home Eksamen DM500 Efterår 2020}

\author{
\\
h8 - studiegruppe 3:
\\
Andreas Rosenstjerne Hansen(andrh20)
\\
Frederik Mortensen Dam(Frdam20)
\\
Gabrielle Hvid Benn Madsen (gamad20)
\\
Nanta Veliovits (navel16)
\\
}

\date{November 2020}

\begin{document}
\maketitle
\pagebreak

\section{Opgave 1(Reeksamen februar 2015)}
I det følgende lader vi $U = \{ 1,2,3,...,15 \} $ være universet (universal set).
\\
\newline Betragt de to mængder
\newline
\begin{math}
A = \{ 2n|n \} \in S og B \{ 3n + 2|n \} \in S
\end{math}
\newline hvor $S$= \{1,2,3,4\}
\\
\newline Angiv samtlige elementer i hver af følgende mængder
\\
\newline
a)\;{$A$\:\{2,\:4,\:6,\:8\}}
\\
\newline
b)\;{$B$\:\{5,\:8,\:11,\:14\}}
\\
\newline
c)\;$A \cap B \{ 8 \} $
\\
\newline
d)\; $A \cup B \{2,\:4,\:5,\:6,\:8,\:11,\:14\} $
\\
\newline
e)\;$A$\:-\:$B$\:\{2,\:4,\:6\}
\\
\newline
f)\; $\overline{\rm A}\;\{\:1,\:3,\:5,\:7,\:9,\:10,\:11,\:12,\:13,\:14,\:15\} $

\section{Opgave 2 (Reeksamen februar 2015)}
a)\; Hvilke af følgende udsagn er sande?

\begin{align*}
\forall x \in N:\exists\:y \in N: x< y \;\;\;Det\;er\;sandt.
\end{align*}
\begin{align*}
\forall x \in N:\exists\;!\:y \in N: x< y \;\;\;Det\;er\;ikke\;sandt.
\end{align*}
\begin{align*}
\exists y \in N:\:\forall x \in N: x< y \;\;\;Det\;er\;sandt.
\end{align*}

b)\;Angiv negeringen af udsagn 1. fra spørgsmål a).\\\newline Negerings-operatoren (¬) må ikke indgå i dit udsagn.
 \begin{align*}
\exists\: x \in N:\forall\:y \in N: x > y
\end{align*}

\section{Opgave 3 (Reeksamen februar 2015)}
Lad R, S og T være binære relationer på mængden \{1, 2, 3, 4\}.

a) Lad R = \{(1,1), (2,1), (2,2), (2,4), (3,1), (3,3), (3,4), (4,1), (4,4)\}. Er R en partiel ordning?
\\
\newline
Svar: Ja, da relationen er både refleksiv, antisymmetrisk og transitiv.
\\
\newline
b) Lad S = \{(1,2), (2,3), (2,4), (4,2)\}. Angiv den transitive lukning af S.
\\
\newline
Svar: Vi tilføjer de par, der mangler for, at relationen lever op til den transitive egenskab. Den nye relation ser sådan ud: \{(1,2), (1,3), (1,4), (2,2), (2,3), (2,4), (4,2), (4,3), (4,4)\}
\\
\newline
c) Lad T = \{(1,1), (1,3), (2,2), (2,4), (3,1), (3,3), (4,2), (4,4)\}. Bemærk at T er en ækvivalensrelation. Angiv T's ækvivalensklasser.
\\
\newline
Svar:
\newline
$ [1] \cup [3] = \{ 1,3 \} $
\\
\newline
$ [2]\cup[4] = \{2,4\} $
\\
\newline
Opskriv desuden matricerne, der repræsenterer de tre relationer R, S og T.
\\
\newline
\begin{math}
R =
\begin{bmatrix}
1 & 0 & 0 & 0\\
1 & 1 & 0 & 1\\
1 & 0 & 1 & 1\\
1 & 0 & 0 & 1
\end{bmatrix}
\end{math}
\\
\\
\newline
T =
\begin{math}
\begin{bmatrix}
1 & 0 & 1 & 0\\
0 & 1 & 0 & 1\\
1 & 0 & 1 & 0\\
0 & 1 & 0 & 1
\end{bmatrix}
\end{math}
\\
\\
\newline
S =
\begin{math}
\begin{bmatrix}
0 & 1 & 0 & 0\\
0 & 0 & 1 & 1\\
0 & 0 & 0 & 0\\
0 & 1 & 0 & 0
\end{bmatrix}
\end{math}

\section{Opgave 1 (Reeksamen Januar 2012)}
Betragt funktionerne $f$ : $\mathbb{R}$ → $\mathbb{R}$ og $g$ : $\mathbb{R}$ → $\mathbb{R}$ defineret ved
\\
\begin{align*}
f(x) = x +2 + x + 1
\\
g(x) = 2x - 2
\end{align*}
a) Er $f$ en bijektion?
\\
\\
Svar: Nej, da vi ved $x = 1$ og $x = -2$ får resultatet $3$
\\
\\
\newline
b) Har $f$ en invers funktion?
\\
\\
Svar: Nej, da det ikke er en bijektion, da den skal opfylde $f(x) = y$ og $f^-1(y) = x$
\\
\\
c) Angiv $f + g$.
\\
\\
Svar: Vi sætter  $f(x) + g(x)$ sammen
\\
\begin{align*}
f(x) + g(x) = x^2 + x +1 + 2x - 2
\\
\Updownarrow
\\
2x^3 + 3x
\end{align*}
a) Angiv $g \circ f$
\\
\\
Svar:
Vi sætter $f(x)$ ind i $g(x)$, så vi får
\begin{align*}
g(f(x)) = 2(x^2 + x + 1)-2
\\
\Updownarrow
\\
2x^2 + 2x
\end{align*}

\end{document}
