\documentclass{article}
\usepackage[utf8]{inputenc}
\usepackage{blindtext}
\usepackage[T1]{fontenc}
\usepackage{amsmath}
\usepackage{graphicx}
\usepackage{listings}
\usepackage{color} 

\title{Take-Home Eksamen DM500 Efterår 2020}

\author{
\\
h8 - studiegruppe 3:
\\
Andreas
\\
Frederik
\\
Gabrielle Hvid Benn Madsen (gamad20)
\\
Nanta Veliovits (navel16)
\\
}

\date{November 2020}

\begin{document}
\maketitle
\pagebreak

\section{Opgave 1(Reeksamen februar 2015)}\
I det følgende lader vi\; 
$U$ = \{1,2,3,...,15\} være universet (universal set).
\\
\newline Betragt de to mængder
\newline
\begin{align*}{$A$=\{2n|n \}\in S\;og\;$B$\{3n + 2|n\}\in S\}}
\end{align*}
\newline hvor $S$= \{1,2,3,4\}
\\
\newline Angiv samtlige elementer i hver af følgende mængder
\\
\newline
a)\;{$A$\:\{2,\:4,\:6,\:8\}}
\\
\newline
b)\;{$B$\:\{5,\:8,\:11,\:14\}}
\\
\newline
c)\;$A$\:\cap\:$B$\:\{8\}
\\
\newline
d)\;$A$\:\cup\:$B$\:\{2,\:4,\:5,\:6,\:8,\:11,\:14\}
\\
\newline
e)\;$A$\:-\:$B$\:\{2,\:4,\:6\}
\\
\newline
f)\:\overline{\rm A}\;\{\:1,\:3,\:5,\:7,\:9,\:10,\:11,\:12,\:13,\:14,\:15\}}

\section{Opgave 2 (Reeksamen februar 2015)}
a)\;Hvilke af følgende udsagn er sande?
\begin{enumerate}
\begin{align*}
\forall x \in N:\exists\:y \in N: x< y \;\;\;Det\;er\;sandt.
\end{align*} 
\begin{align*}
\forall x \in N:\exists\;!\:y \in N: x< y \;\;\;Det\;er\;ikke\;sandt.
\end{align*} 
\begin{align*}
\exists y \in N:\:\forall x \in N: x< y \;\;\;Det\;er\;sandt.
\end{align*} 
\end{enumerate}
\newline
b)\;Angiv negeringen af udsagn 1. fra spørgsmål a).\\\newline Negerings-operatoren (¬) må ikke indgå i dit udsagn.
 \begin{align*}
\exists\: x \in N:\forall\:y \in N: x > y 
\end{align*} 
\end{document}